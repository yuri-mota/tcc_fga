%##############################################%
% aspectosgerais < Desenvolvimento             %
% consideracoes < Metodologia                  %
% textoepostexto < Resultados                  %
% elementosdotexto < Conclusão                 %
%##############################################%


\chapter[Conclusão]{Conclusão}

	Os resultados apresentam um número notável de exceções de segurança, sendo disparadas pela verificação realizada pelo \textit{OpenSSL}. Estas, encontradas em ambiente controlado de estudo, mostram que existem falhas em certificados obtidos de um ambiente real, um dado preocupante, visto que todos os certificados apresentados deveriam ser seguros. Essas exceções podem se tornar um risco de falhas maiores de segurança, em ambiente real, resultando em riscos na comunicação, que supostamente deveria ser segura.

	Um número percentual de exceções pode ser notado no corpo de certificados estudados, entretanto não é possível afirmar uma única causa para todos os erros. Foi observado que um grande número de certificados apresentaram o \textit{Erro 02} ao serem analisados e isso serve como um forte indicador de que a origem dos problemas pode estar na \textit{cadeia de assinaturas digitais} que se faz presente em todo certificado digital. Essa cadeia, que deveria ser confiável, pode não estar funcionando de forma ideal, e suas razões devem ser investigadas.

	Dentro desse contexto, os certificados que apresentaram o \textit{Erro 20} também foram numerosos, e sua natureza leva a crer que os mesmos riscos advindos do \textit{Erro 02} podem se repetir nesse cenário. Isso leva a crer que existe uma grande vulnerabilidade, atualmente, em como está sendo tratada a cadeia de certificação e suas referências. A própria ideia da razão dos certificados digitais dependeria dessa cadeia, que é a justificada na confiabilidade através de terceiros.

	A presença do \textit{Erro 10} no estudo pode apresentar casos circunstanciais, em que a falta de um controle mais rígido levou à repetição da detecção deste erro dentro do estudo. A natureza do erro e sua incidência não se apresentam de forma tão alarmante, seria necessário analisar pontualmente para entender se os casos são maliciosos.

	Já o \textit{Erro 18}, diferente do \textit{Erro 10} de recorrência similar, pode apresentar casos mais perigosos, onde a brecha de segurança seria mais facilmente explorada. Esses casos, possuem suas exceções, mas a natureza das exceções deixa claro que sua recorrência não deveria se mostrar como um valor tão significativo quanto foi visto.

	Os \textit{erros} retornados em menor número durante as verificações podem indicar casos isolados. A ocorrência desses erros também é preocupante, pois podem acarretar em problemas de segurança, se caracterizando como falhas dependendo de como forem explorados. Isso deixa claro que erros de segurança existem, e podem ser encontradas em certificados em uso, um fator alarmante, uma vez que o uso de certificados digitais é cada vez mais um recurso explorado como forma de aumentar a segurança nas transações realizadas por vias de comunicação digitais.
