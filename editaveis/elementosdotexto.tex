%##############################################%
% aspectosgerais < Desenvolvimento             %
% consideracoes < Metodologia                  %
% textoepostexto < Resultados                  %
% elementosdotexto < Conclusão                 %
%##############################################%


\chapter[Conclusão]{Conclusão}

	Os resultados apresentam um número notável de exceções de segurança, sendo disparadas pela verificação realizada pelo \textit{OpenSSL}. Estas, encontradas em ambiente controlado de estudo, mostram que existem falhas em certificados obtidos de um ambiente real, um dado preocupante, visto que todos os certificados apresentados deveriam ser seguros. Essas exceções podem se tornar um risco de falhas maiores de segurança, em ambiente real, resultando em riscos na comunicação, que supostamente deveria ser segura.

	Um número percentual de exceções pode ser notado no corpo de certificados estudados, entretanto não é possível afirmar uma única causa. Foi observado que um grande número de certificados apresentaram o \textit{Erro 20} ao serem analisados e isso serve como um forte indicador de que a origem dos problemas pode estar na \textit{cadeia de assinaturas digitais} que se faz presente em todo certificado digital. Essa cadeia, que deveria ser confiável, pode não estar funcionando de forma ideal, e suas razões devem ser investigadas.

	Outras \textit{exceções} foram retornadas em menor número durante as verificações, o que pode indicar erros isolados. A ocorrência desses erros também é preocupante, pois essas exceções também podem acarretar em problemas de segurança, que podem se caracterizar como falhas dependendo de como forem explorados. Isso indica que as exceções de segurança existem, e podem ser encontradas em certificados em uso, um fator alarmante, uma vez que o uso de certificados digitais é cada vez mais um recurso explorado como forma de aumentar a segurança nas transações realizadas por vias de comunicação digitais.
