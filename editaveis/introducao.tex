%%%%%%%%%%%%%%%%%%%%%%%%%%%%%%%%%%%%%%%%
%Introdução
%-Pré-Projeto {TCC/Artigo}
%Problema
%Justificativa
%>Hipótese/Questão/Pergunta
%Objetivo >Geral >Específico
%Metodologia
%Descrever o trabalho em frases < Desenvolvimento
%%%%%%%%%%%%%%%%%%%%%%%%%%%%%%%%%%%%%%%%

\chapter[Introdução]{Introdução}
	Certificados digitais são uma forma comum, atualmente, de comprovar a identidade de pessoas, físicas ou jurídicas, em meios digitais. Essa ferramenta de identificação precisa ser robusta, do contrário colocaria em dúvida a confiabilidade do sistema de identificação digital, entretanto, essas falhas existem e dificilmente são tratadas com a devida importância. Para garantir a segurança dos usuários de certificação digital, é necessário expor e identificar tais falhas.

	Pela da exposição das falhas mais comuns entre os certificados digitais avaliados, nesse estudo, realiza-se uma análise acerca de quais fatores levam a essas falhas, se elas podem ser sanadas, e quais as principais implicações ocasionadas pela ocorrência das mesmas.

\section[Objetivo Geral]{Objetivo Geral}

	Este trabalho se foca na identificação, por amostragem em sites que utilizam certificação digital, de quais são as principais vulnerabilidades de segurança causadas por certificados digitais que apresentam erros, utilizados em ambiente real.

\section[Objetivos Específicos]{Objetivos Específicos}

	Por estes mesmos meios, se busca:
	\begin{itemize}
		\item Conhecer as estruturas internas dos certificados digitais utilizados em ambiente real; 
		\item Elucidar formas de contornar exceções de segurança causadas por problemas em certificados digitais.
	\end{itemize}

\section[Metodologia]{Metodologia}
	
	Dentro do padrão X.509, os tipos de erros serão divididos em grupos para que a análise esclareça quais as formas pelas quais esses erros podem se apresentar, e assim seja possível entender a origem e razão de tais erros. Todos os grupos de erros serão originários, e baseados, dos erros encontrados pela verificação realizada com a ferramenta \textit{OpenSSL}, uma ferramenta aberta utilizada para realizar esse tipo de verificação.

\section[Organização do Trabalho]{Organização do Trabalho}
	
	Na primeira parte deste trabalho, é apresentada uma introdução à teoria envolvida na criptografia, tendo como foco as razões pelas quais ela se faz necessária, os tipos de ameaças digitais conhecidas, e as medidas tomadas para contornar esses problemas.
	
	Uma vez que esses detalhes estão esclarecidos, será discutida a forma como a criptografia age para possibilitar a existência, e uso, dos certificados digitais e os algoritmos envolvidos em sua aplicação.
	
	Essa informação servirá para descrever as características que permeiam os certificados digitais, e que permitem que eles sejam considerados como seguros em um ambiente real de aplicação, onde eles devem carregar informações suficientes para atestar a identidade de um indivíduo, ou grupo, em meio digital.
